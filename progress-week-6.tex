\section{Introduction}
I have spent the majority of my time this week grasping the key ideas of Generative Adversarial from CVPR-2017\cite{cvpr17} and ICCV-2017\cite{iccv17}. 

\section{CVPR-2017}
Lecture video could be found \href{https://www.youtube.com/watch?v=KudkR-fFu_8}{here}.

This lecture was taught by MingYu Liu, Jan Kautz and Julie Bernauer. Because the sound quality was so bad and this lecture was aimed at people that already have deep knowledge about what generative adversarial network is, so it was real hard for me to understand what they was saying.

I only captured a few things about generative adversarial network and all of my taken note could be found here on my \href{https://github.com/tlvu2697/collection--generative-adversarial-network/tree/master/cvpr-17}{GitHub}.

\section{ICCV-2017}
Lecture videos could be found \href{https://www.youtube.com/playlist?list=PLazcgz-LJ6ZIrJV-qiqw16JyJFuE4TNKF}{here}.

This was a combination of multiple lectures:
\begin{enumerate}
\item Introduction to GAN\cite{gan}
\item Autoencoder GANs
\item Conditional GANs\cite{cgan}, StackGAN\cite{stackgan}, StackGAN v2\cite{stackganv2}
\item GANs in the Wild
\item Evaluating Generative Models
\item Domain Adversarial Learning\cite{adda}\cite{rozantsev}
\item Visual Synthesis and Manipulation with GANs
\item Do GANs learn the distribution?
\item Connections between adversarial training and RL
\item GANs as Learned Loss Functions
\end{enumerate}

All of my taken note could be found here on my \href{https://github.com/tlvu2697/collection--generative-adversarial-network/tree/master/iccv-17}{GitHub}.